
\documentclass{article}


\usepackage[a4paper, inner=1.7cm, outer=2.7cm, top=2cm, bottom=2cm, bindingoffset=1.2cm]{geometry}
\usepackage[english]{babel}
\usepackage{graphicx}
	\graphicspath{ {images/} }
\usepackage{wrapfig}
\usepackage{paralist}
\usepackage{blindtext}
\usepackage{enumitem}
\usepackage{fancyhdr}
\usepackage{amsmath}
\usepackage[utf8]{inputenc}
\usepackage[toc,section=section]{glossaries}
\usepackage{enumitem}
\usepackage{lipsum}
\usepackage{amsmath}

\begin{document}


\subsection{Strategy}

In order to develop a working software calculator prototype in the coming weeks, our team has come up with a development strategy to be prepared for challenges that we may face. This strategy must take into account a plan for writing requirements for features implemented, the technologies selected to develop the calculator, ideas of algorithms for numerical computation of selected functions, and tasks to be allocated to each team member based each of our strengths. With a well advised plan of action, we are much more likely to collaborate efficiently as a team and ultimately meet the deadline of the project. 


\subsection{Choice of technology}

Our team needed to select technologies to meet our software needs while also complementing the team members’ development expertise. The necessary technologies included:
\begin{compactitem}
\item a main programming language with unit testing libraries, a code documentation interface and a graphical user interface library
\item communication tools
\item version control software
\end{compactitem}

For our primary programming language we decided to pick the language in which all of our team members had experience writing code in, Java. This allowed each team member to jump right into coding when the time came since no one was blocked having to learn a new language. Java is also advantageous since it has many libraries that we would be able to use for unit testing (Junit), GUI development (JavaFX) and code documentation (Javadoc). Additionally, its object-oriented design fit with how we envisioned designing our calculator program. 

In order to coordinate with each other outside of regular meeting hours, we decided to use Discord as our communication tool. Discord can be accessed on almost any device, is very reliable and is easy to use. It also allowed us to create different chat channels for different subjects in order to keep our discussions on topic. For example, one channel could be for scheduling and one could be for brainstorming ideas about algorithms. 

We decided to choose git as our version control software since it is simple enough to use, it has tons of documentation to support our needs and some of our team had already used it. Furthermore, they would allow us to work on the same files at different times and easily keep track of changes made to any of the files. 


\end{document}



