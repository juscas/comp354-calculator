
% Dieser Text ist urheberrechtlich geschützt
% Er stellt einen Auszug eines von mir erstellten Referates dar
% und darf nicht gewerblich genutzt werden
% die private bzw. Studiums bezogen Nutzung ist frei
% April 2005 
% Autor: Sascha Frank 
% Universität Freiburg 
% www.informatik.uni-freiburg.de/~frank/

\documentclass[12pt,twoside,a4paper]{article}
\begin{document}

\section{Warum diese Dokumentation?}

Diese Dokumentation habe ich f\"ur den \LaTeX \ Kurs vom 19.04.2005 
bzw. 21.04.2005 
gemacht, und sie soll den Kursteilnehmer bzw. anderen Interesierten,
ein groben \"Uberblick geben. 

\section{Was bisher gezeigt wurde}

Hier erst einmal ein grober \"Uberlick was bisher gezeigt wurde. 

\subsection{sehr kleine \& kleine \LaTeX \ Dokumente}

Hierbei ging es mir vorallem darum zu zeigen, da\ss \ man keine
20 zeilige preamble, d.h. der Text der vor dem eigentlichen 
Dokument kommt, bedarf um ein \LaTeX \ Dokument zu erstellen. \\
Leider hab ich viele alte Vorlagen gesehen in denen da\ss \ der Fall war.

\subsection{Tabellen} 

\"Uber Tabellen in \LaTeX \ gibt es so viel zu sagen, da\ss \ man damit ganze 
B\"ucher f\"uhlen k\"onnte bzw. dies auch gesehen ist. 

\subsection{\"Ubungsbl\"atter}
Mal ein praktischer Einstieg in \LaTeX . Es gibt zum Teil, den Zwang seine 
\"Ubungsaufgaben in pdf bzw. ps Format in elektronischer Form abzugeben.
Und je nach Fach ist dabei \LaTeX \ anderen Produkten vorzuziehen vv.

\section{Ausblick}

So wie es aussieht wird es noch ein paar Fortsetzungskurse geben,
so da\ss \ ich in der n\"achsten Zeit mehr als genug zu erledigen habe.  
  



\end{document}