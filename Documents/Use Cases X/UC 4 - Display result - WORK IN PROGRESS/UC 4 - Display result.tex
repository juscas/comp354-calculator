\documentclass{article}

\usepackage{enumitem}
	\setlist{nolistsep,leftmargin=*}

\begin{document}

\begin{table}[!h]
\begin{tabular}{|p{3cm}|p{9cm}|}
\hline
\textbf{ID} & UC 4  \\ \hline
\textbf{Name} & Display Result  \\ \hline
\textbf{Description} & User wants clear display of the calculation as it is being inputted. User also wants the calculator to display clear results and appropriate error messages. \\ \hline
\textbf{Pre-condition} &
	\begin{itemize}
		\vspace{-2mm}
		\item Result was calculated, or
		\item User inputs values in calculator
		\vspace{-3.5mm}
	\end{itemize}  \\ \hline
\textbf{Post-condition} & 
	\begin{itemize}
		\vspace{-2mm}
		\item Intermediate calculation is displayed on screen.
		\item Final results are displayed on screen.
		\item Error messages are displayes correctly.
		\vspace{-3.5mm}
	\end{itemize}  \\ \hline
\textbf{Basic path} &
	\begin{enumerate}
		\vspace{-2mm}
		\item User turns on the calculator
			\begin{itemize}
			\item[1a.] Welcome message is displayed.
			\end{itemize}
		\item User enters a 
		\vspace{-3.5mm}
	\end{enumerate}  \\ \hline
\textbf{Alternative Path} &
	\begin{itemize}[leftmargin=6mm]
		\vspace{-2mm}
		\item [1b.] sample
		\item [3a.] Calculator detects a syntax or arithmetic error in user's input.
			\begin{enumerate}
				\item Calculator detect the type of exception.
				\item Calculator displays this exception on screen (UC 4)
				\item User can clear exception and return to the offending arithmetic expression and attempt to correct the error (return to Basic Path 2).
			\end{enumerate}
		
		\vspace{-3.5mm}
	\end{itemize}  \\ \hline
\end{tabular}
\caption{UC 4 - Calculate result}
\end{table}



\end{document}