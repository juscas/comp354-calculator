\documentclass{article}

\begin{document}

\subsection{Source code review}

This iteration was mostly marked by requirements gathering mainly through the interview process. A limited amount of research was done into evaluating different algorithms for implementing of some functions. \\

While that was ongoing, we thought that we needed something concrete to present at this stage. While nothing look even remotely like a calculator at this stage, several function were implemented satisfactorily. \\

For example, $ln(x)$, $sin(x)$, $e^x$, and $\sqrt[n]{x}$ were implemented with some measure of success. Choices were made to used the simplest algorithm at this stage and to only test over a certain range of inputs. The full implementation will be done for the next iteration. \\

Following the interviews with the engineering students, we decided on implementing some extra functionality such as $x!$, binomial coefficient, etc. to better satisfy this class of users's needs. We were able to justify their inclusion at this early stage in part due to their relatively simple implementations. \\

There are still an immense amount of implementation details that the team needs to agree on at this stage. The interview process gave us so many ideas and potential requirements/features that we have not had enough time to debate on. We reserve the resolution of these issues for the next iteration. At this stage, it is exciting to see how the product is taking shape. With every meeting we feel we are getting tangibly closer to the final version of the product. We know that we will have some hard decisions to make very soon and that there may be some slight head butting. Regardless, we are all keen on compromising and getting a good product out the door.

\end{document}