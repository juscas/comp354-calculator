\documentclass{article}

\usepackage{enumitem}
%	\setlist{nolistsep,leftmargin=*} % this breaks the final report's other enumerates - don't know why?? Its awesome formating if you can get it to not break everything though!!

\begin{document}



\begin{table}[!h]
\begin{tabular}{|p{3cm}|p{9cm}|}
\hline
\textbf{ID} & UC 1  \\ \hline
\textbf{Name} & Calculate result  \\ \hline
\textbf{Description} & User wants the calculator to resolve his mathematical expression to a satisfactory level of precision near-instantaneously.  \\ \hline
\textbf{Pre-condition} &
	\begin{itemize}
		\vspace{-2mm}
		\item Calculator is on
		\vspace{-3.5mm}
	\end{itemize}  \\ \hline
\textbf{Post-condition} &
	\begin{itemize}
		\vspace{-2mm}
		\item Calculator takes user input, parses, calculates, and arrives at a correct result.
		\item Calculator saves the result of this operation for future use (UC 2).
		\vspace{-3.5mm}
	\end{itemize}  \\ \hline
\textbf{Basic path} &
	\begin{enumerate}
		\vspace{-2mm}
		\item This use case starts with the user entering a mathematical expression into the calculator.
		\item When satisfied with inputted expression user presses "=" button or "enter" on keyboard.
		\item Calculator performs resolution of the arithmetic expression.
		\item Calculator stores the result of the expression (UC 4).
		\vspace{-3.5mm}
	\end{enumerate}  \\ \hline
\textbf{Alternative Path} &
	\begin{itemize}[leftmargin=6mm]
		\vspace{-2mm}
		\item [1b.] User enters a letter and number in order to store a variable (UC 3).
		\item [3a.] Calculator detects a syntax or arithmetic error in user's input.
			\begin{enumerate}
				\item Calculator detect the type of exception.
				\item Calculator displays this exception on screen (UC 4)
				\item User can clear exception and return to the offending arithmetic expression and attempt to correct the error (return to Basic Path 2).
			\end{enumerate}
		
		\vspace{-3.5mm}
	\end{itemize}  \\ \hline
\end{tabular}
\caption{UC 1 - Calculate result}
\end{table}



\end{document}