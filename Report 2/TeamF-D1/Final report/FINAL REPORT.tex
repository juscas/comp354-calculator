\documentclass[a4paper]{article}

\usepackage[a4paper, inner=1.7cm, outer=2.7cm, top=2cm, bottom=2cm, bindingoffset=1.2cm]{geometry}
\usepackage[english]{babel}
\usepackage{graphicx}
	\graphicspath{ {images/} }
\usepackage{wrapfig}
\usepackage{paralist}
\usepackage{blindtext}
\usepackage{enumitem}
\usepackage{fancyhdr}
\usepackage{amsmath}
\usepackage[utf8]{inputenc}
\usepackage[toc,section=section]{glossaries}
\usepackage{enumitem}

\makenoidxglossaries
 
\newglossaryentry{java}
{
    name=Java,
    description={A general purpose object-oriented programming language}
}

\newglossaryentry{javafx}
{
    name=JavaFX,
    description={A graphical user interface library for Java}
}

\newglossaryentry{git}
{
    name=Git,
    description={An open sourece distributed version control software}
}

\newglossaryentry{github}
{
    name=GitHub,
    description={A web service that hosts git repositories for ease of use between developers}
}

\newglossaryentry{discord}
{
    name=Discord,
    description={A communication software hosted on the web used for scheduling, discussion and sharing of files}
}

\newglossaryentry{junit}
{
    name=Junit,
    description={A library in Java used for writing unit tests}
}

\newglossaryentry{latex}
{
    name=Latex,
    description={A textual interface for writing technical and scientific documents}
}

\newglossaryentry{javadoc}
{
    name=JavaDoc,
    description={A code documentation generator for Java}
}

\newglossaryentry{transcendentalfunction}
{
    name=Transcendental Function,
    description={"A function that does not satisfy any single-variable
polynomial equation whose coefficients are themselves roots of polynomials"}
}

\newglossaryentry{eclipse}
{
    name=Eclipse,
    description={A Java Integrated Development Environment}
}

\newglossaryentry{intellij}
{
    name=IntelliJ,
    description={A Java Integrated Development Environment}
}


\makeindex

\setlength{\parindent}{0pt}

\begin{document}


% TITLE PAGE CODE ------------------------------------
\title{\LARGE{\textbf{Team F - ETERNITY Calculator}}}
\author{
	Castonguay, Justin | Fakhr, Daniel | Hernandez, Jaime Andres \\ Thibault-Shea, Daniel |
	Yaghma, Ashkhan \\
}
\date{August 5, 2019}

\fancyhf{}

\clearpage\maketitle
\thispagestyle{empty} % Ensures no page numbering on title page
\pagebreak

\setcounter{page}{2} % Start counting on page 2
\fancyhf{}
\renewcommand{\headrulewidth}{2pt}
\renewcommand{\footrulewidth}{1pt}
\fancyhead[LE,RO]{\rightmark}
\tableofcontents
\pagebreak


\section{Introduction and Lead Up to Iteration 2}

\subsection{Recap of Iteration 1}

\subsection{Intro to current Iteration}

\subsection{Goals for current Iteration}

\section{Interviews of Potential Users}

\section{Personas Derived from Interviews}

\section{Use Cases}

\section{Design}

\subsection{Influence of Use Cases on Design Decisions}

\subsection{Macro Architecture}

\subsection{Micro Architecture}

\subsection{GUI Design and User Experience}

\section{Implementation}

\subsection{Roadblocks During Implementation}

\subsection{Adjustment to the Micro Architecture}

\subsection{Outstanding Issues}

\section{Testing}

\subsection{Usefulness of Debugger in Testing}

\subsection{Unit Testing}

\subsection{Integration Testing}

\subsection{Acceptance Testing}

\section{Varia}

\section{Appendices}

\subsection{Use of a Debugger}

\subsection{ASQ Questions}

\subsection{Calculator Manual}

\end{document}

