\documentclass{article}

\usepackage{enumitem}
	%\setlist{nolistsep,leftmargin=*}




\begin{document}



\begin{table}[!h]
\begin{tabular}{|p{3cm}|p{9cm}|}
\hline
\textbf{ID} & UC 3  \\ \hline
\textbf{Name} & Store variables  \\ \hline
\textbf{Description} & User want to store values that can be recalled during calculations by referencing an alphabetical label.  \\ \hline
\textbf{Pre-condition} &
	\begin{itemize}
		\vspace{-2mm}
		\item Calculator is on
		\vspace{-3.5mm}
	\end{itemize}  \\ \hline
\textbf{Post-condition} &
	\begin{itemize}
		\vspace{-2mm}
		\item A number is stored in the calculator's memory and is ready to be retrieved by invoking its alphabetical label.
		\item User should be able to clear or overwrite a stored variable.
		\vspace{-3.5mm}
	\end{itemize}  \\ \hline
\textbf{Basic path} &
	\begin{enumerate}
		\vspace{-2mm}
		\item This use case starts with the user entering an alphabetical label that will eventually be used to recall the stored value.
		\item The user then presses "equals" to indicate that a value is to be stored under the chosen label.
		\item The user then presses "enter" which tells the calculator to store the variable under the aforementioned label.
		\item At any point during a calculation (UC 1), the user can evoke the value stored in a variable by entering the corresponding alphabetic character.
		\item The calculator substitutes the variables's value into the calculation.
		\vspace{-3.5mm}
	\end{enumerate}  \\ \hline
\textbf{Alternative Path} &
	\begin{itemize}[leftmargin=6mm]
		\vspace{-2mm}
		\item [1a.] Clearing the variable
			\begin{enumerate}
				\item User enters the alphabetic label of the variable that requires clearing (value and label appear on display).
				\item User presses "clear".
				\item The calculator shows the variable is now cleared.
			\end{enumerate}
		\vspace{-3.5mm}
	\end{itemize}  \\ \hline
\end{tabular}
\caption{UC 3 - Store variables}
\end{table}



\end{document}