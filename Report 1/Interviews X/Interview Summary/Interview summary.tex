\documentclass{article}

\usepackage[a4paper, inner=1.7cm, outer=2.7cm, top=2cm, bottom=2cm, bindingoffset=1.2cm]{geometry}
\usepackage[english]{babel}
\usepackage{graphicx}
	\graphicspath{ {images/} }
\usepackage{wrapfig}
\usepackage{paralist}
\usepackage{blindtext}
\usepackage{enumitem}
\usepackage{fancyhdr}
\usepackage{amsmath}
\usepackage[utf8]{inputenc}
\usepackage[toc,section=section]{glossaries}
\usepackage{enumitem}

\begin{document}

\subsection{Common interviewee points summarized}

\textbf{\ \ \ \  Sarah}

\underline{HR Student}

\begin{compactitem}
\item inputting full eqn like she memorized
\item physical calculator 
\item simple calculator
\item prefers physical calculator but uses others
\item functions in the book
\item downloadable functions or packages
\item cares about precision and the right answer
\item doesn’t care about aesthetics
\item hot keys for common functions
\end{compactitem}
\bigskip

\textbf{Victoria  Benlala}

\underline{Entrepreneur, Spa Owner}

\begin{compactitem}
\item Button to calculate the taxes (simple programmable functions)
\item phyiscal calculatior first doesn’t mind others
\item doesn’t use complicated functions
\item simplicity and soft buttons. Would like a more portable version.
\item mapping numbers to number keys on computer. Being able to have hot keys or set them up himself with the functions he or she is given
\end{compactitem}
\bigskip

\textbf{Kevin}

\underline{Engineering student}

\begin{compactitem}
\item Would like to be able to access functions easily for engineering 
\item Comfortable with both software and hardware but prefers hardware
\item He wants shortcuts
\item Wants basic functions also
\item Wants to use computer keyboard and not mouse pointer
\item Portable and key mappable
\item Use symbols that are already commonly found on calculator on the cpu keyboard also 
\item Include a shortcut quit key
\item Hot keys (like S for sin, T for Tan, etc.)
\item Recommends skins for calculator
\item Would like downloadable packages for functions to customize calculator
\end{compactitem}
\bigskip

\textbf{Tarek}

\underline{Electrical engineering student}

\begin{compactitem}
\item accuracy, speed, and comfort
\item basic essential functions
\item he would like it to be able to plot graphs 
\item he would like to transfer his work from calculator to mobile
\item prefers physical but he uses other for quick calculations
\item wants calculator easy to hold
\item would like it to be cheap even if it’s customizable
\end{compactitem}
\bigskip

\textbf{Arash}

\underline{Avionics Engineering Student}

\begin{compactitem}
\item specific buttons for each function
\item prefers an app 
\item He would assign each function to a specific button
\end{compactitem}


\subsection{Common Ideas}
\begin{itemize}
\item Simplicity 
\item Physical calculators $\rightarrow$ GUI could look like physical calculator
\item Hot keys 
\item Simple functions (plus, minus, etc.) 
\item Portability 
\item Customizable (physical and software wise)  download functions 
\end{itemize}

\subsection{Summary}

Looking through all the interviews we were able to pick up on some important points that we chose to consider when creating our use case diagrams and to move forth with our project. We interviewed a Human Resource, Mechanical Engineering, Electrical Engineering, and Avionics Engineering student. We also interviewed an Entrepreneur/Spa Owner to gather our data. Each individual had very different needs specifying what kinds of functions they would like to see on their ideal calculator. For example, Engineers wanted integration functions, while an entrepreneur wanted percentages or tax calculating functions. \\

What they all had in common though was the want for simple operations (like addition, subtraction, etc.). They all also wanted simplicity in terms of the calculator's look, how easy it would be to access the functions they wanted to use, understand what they are, and it’s portability. The majority also wanted a reliable calculator in terms of precision and accuracy. They all preferred physical calculators over software calculators (like those you would find on a computer as an extra tool application). They all liked the idea of mapping keys on a computer keyboard to their desired functions to make the calculating experience more personalized and simple to them. They all had different ways they wanted to customize their calculator, which included personalizing it physically and software-wise. The important point though was that customizability was what they valued commonly amongst each other. \\

Based on the research we made from the stakeholders we interviewed, the calculator will need to be \textbf{simple}, \textbf{customizable}, and \textbf{reliable}.


\end{document}